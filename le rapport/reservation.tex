\documentclass{article}
\usepackage[utf8]{inputenc}
\usepackage{graphicx}
\usepackage[dvipsnames]{xcolor}
\usepackage{fourier}
\definecolor{amethyst}{rgb}{0.6, 0.4, 0.8}
\title{Reservation des salles}
\author{Mohammed AL JADD}
\date{2020 Avril}
\begin{document}
\maketitle

    



\includegraphics[width=\textwidth]{img/Go.png}

\begin{enumerate}
    
     \item  \textcolor{red}{\huge Introduction} : 
     \vspace{1cm}
 
   \setlength{\parindent}{1cm} Mon projet est une application web , sous le nom Réservation des salles à distance, qui répondra aux problématiques suivantes:
   \begin{itemize}
     
        \item Difficulté à trouver une salle disponible.
        \item Le professeur a besoin d'être dans l'institut pour trouver une salle disponible.
        \item L'emploi du temps doit être modifié chaque fois qu'il n'y a pas de salle disponible.
        
            Les langages de programmation que j'ai utilisés dans mon projet : \textcolor{blue}{ \textbf{HTML}},\textcolor{blue}{\textbf{CSS}},\textcolor{blue}{\textbf{JavaScript}} et \textcolor{blue}{\textbf{PHP}}.
   \end{itemize}
   
   Il m'a fallu environ deux mois pour terminer ce projet. J'ai commencé en février et terminé en avril.\par Ce projet m'a permis d'apprendre à travailler avec github et à écrire du code long en \textcolor{blue}{\textbf{PHP}}, \textcolor{blue}{\textbf{CSS}} et \textcolor{blue}{\textbf{JavaScript}}. De plus, comme j'ai appris certaines commandes \textcolor{blue}{\textbf{GITHUB}}, cela me permettra à l'avenir de collaborer avec d'autres peuples, même à la maison.\par l'une des choses importantes que le projet me fournit est de savoir comment gérer votre projet et le discrétiser en étapes pour faciliter son progression.\par Pour \textcolor{blue}{\textbf{GITHUB}}, la meilleure chose que j'ai aimé est que vous ne serez pas attiré par votre machie puisque votre projet est stocké en ligne.
   
   
   
   
   
   
   \item \textcolor{red}{\huge L'avancement du projet} :
   
    Ci-dessous, je vais expliquer chaque tâche en détail:
        
        \begin{enumerate}
        \item \textcolor{amethyst}{Apprenez le langage de programmation php}:
        
        \vspace{0.4cm}
            \setlength{\parindent}{1cm} J'ai appris la langage de programmation PHP parce que dans la plupart du temps je vais interagir avec la base de données.
            J'ai suivi des cours PHP sur une chaîne YouTube s'appelle mmtuts en regardant des petites videos.     
            
        \item \textcolor{amethyst}{Déterminer les pages que le site Web contiendra}.
            
            \vspace{0.4cm}
                \setlength{\parindent}{1cm} La détermination des pages est une étape très importante car elle vous donne une vision globale de votre website.L'image suivante montre toutes les pages de notre site Web et montre également les redirections entre elles.
                
               \vspace{0.6cm}
               \hspace*{-1.05in}
               \noindent\makebox[\textwidth]{\includegraphics[width=\paperwidth,height=12cm]{img/map.jpg}}
                \vspace{0.6cm}
                 \item \textcolor{amethyst}{Installation de la base de données}.
           \vspace{0.4cm}
           
           
            \setlength{\parindent}{1cm}  J'ai installer logiciel xampp. Ce permet de mettre en place un serveur Web local et un serveur de Base de donnée. Après avoir réalisé le modèle entité-association, j'ai créé ma base de données qui est présentée dans l'image suivante :
            
            
            
            \hspace*{-1.05in}
               \noindent\makebox[\textwidth]{\includegraphics[width=\paperwidth,height=12cm]{img/dataBase.png}}
        
        
         \item \textcolor{amethyst}{Créer des maquettes}.
         
         \vspace{0.4cm}
                \setlength{\parindent}{1cm} J'ai créé des maquettes pour mon site Web en utilisant le site Web moqups.com. Ils sont présentés dans le dossier maquettes.
         
         \item \textcolor{amethyst}{Apprenez \textbf{GITHUB}}.
         
         \vspace{0.4cm}
                \setlength{\parindent}{1cm} J'ai regardé un video sur youtube pour apprendre comment commiter du VISUAL STUDIO CODE, créer un project basic-KANBAN.
         
         \item \textcolor{amethyst}{Créer des pages WEB}.
         
         \vspace{0.4cm}
                \setlength{\parindent}{1cm} J'ai commencé à créer des pages Web \textcolor{blue}{\textbf{PHP}} après que je termine une page je crée une page \textcolor{blue}{\textbf{CSS}} pour appliquer les style .Mais j'ai respecté la façon que les informations entre par l'utlisateur vont  être traitées dans des pages des différents. Par exemple utilisateur elle veut changer son mot de passe, son nouveau mot de passe va être traité dans une autre page.
         
         \hspace*{-1.05in}
               \noindent\makebox[\textwidth]{\includegraphics[width=\paperwidth,height=12cm]{img/visual.png}}
         \item \textcolor{amethyst}{Sécuriser le site web}.
         
         \vspace{0.4cm}
                \setlength{\parindent}{1cm} Pour la sécurité de ce site Web il y a plusieurs choses que j'ai fait la première chose est d'empêcher l'accès à certaines pages sans connexion. Une deuxième chose est d'empêcher l'accès à certains car ils sont destinés à l'administrateur même en cas de connexion.
         
         \item \textcolor{amethyst}{Ajouter JavaScript}.
         
         \vspace{0.4cm}
                \setlength{\parindent}{1cm} J'ai ajouté JavaScript à mon site Web pour le rendre plus dynamique, par exemple, lorsqu'il y a des erreurs comme si l'utilisateur n'a pas entré ses informations et soumis, le site Web affichera une erreur indiquant que les fichiers sont vides à l'aide de la boîte d'alerte.
         
         
   
    \end{enumerate}
   \item \textcolor{red}{\huge L'illustration du calendrier du projet} :  
   \vspace{1.2cm}
   
         \hspace*{-0.83in}
        \begin{tabular}{|c | c | c|}
        \hline
          La tâche & date de début & date de fin\\ 
        \hline
        Apprenez le langage de programmation \textcolor{blue}{\textbf{PHP}} & 2020/02/22 & 2020/02/28 \\ 
        \hline
        Déterminer les pages que le site Web contiendra & 2020/02/29 & 2020/02/30 \\ 
        \hline
        Installation de la base de données & 2020/03/15 & 2020/03/20\\
        \hline
        Créer des maquettes & 2020/04/01  & 2020/04/01\\
        \hline
        Apprenez \textcolor{blue}{\textbf{GITHUB}} & 2020/04/02  & 2020/04/03\\
        \hline
        Créer des pages WEB & 2020/04/04  & 2020/04/07\\
        \hline
        Sécuriser le site web & 2020/04/04  & 2020/04/11\\
        \hline
        Ajouter \textcolor{blue}{\textbf{JavaScript}} & 2020/04/06  & 2020/04/12
        \end{tabular}
        
   
   \item \textcolor{red}{\huge La presentation du projet} :  
   \vspace{0.7cm}
   

	Lorsque vous entrez sur le site, la première page qui apparaît est la page de connexion
	
	\hspace*{-0.7in}
               \noindent\makebox[\textwidth]{\includegraphics[width=\paperwidth,height=8.5cm]{img/login.png}}
               
    Si vous cliquez sur le bouton Soumettre et en laissant un champ vide, la boîte d'alerte apparaîtra comme l'image suivante montrant :
   \vspace{0.7cm}
   
\hspace*{-0.7in}

               \noindent\makebox[\textwidth]{\includegraphics[width=\paperwidth,height=8.5cm]{img/login-empty-fileds.png}}

Si vos informations étaient erronées, la boîte d'alerte indiquera:

\vspace{0.7cm}
   
\hspace*{-0.7in}

               \noindent\makebox[\textwidth]{\includegraphics[width=\paperwidth,height=6.5cm]{img/login-wrong-infos.png}}


Si vos informations étaient correctes, et vous êtes un utilisateur normal, vous serez érigé sur la page d'accueil avec la boîte d'alerte indiquant une formule de salutation avec Mr/Mme cela dépend du sexe de l'utilisateur:


\vspace{0.7cm}
   
\hspace*{-0.7in}

               \noindent\makebox[\textwidth]{\includegraphics[width=\paperwidth,height=10cm]{img/home.png}}



   La page d'accueil contient une barre de navigation avec 3 autres liens vers d'autres pages.
La première page est la page de contact comme l'image suivante montrant
  
\vspace{0.7cm}
   
\hspace*{-0.7in}

               \noindent\makebox[\textwidth]{\includegraphics[width=\paperwidth,height=10cm]{img/contact.png}}  
               
               Les messages d'alerte sont les mêmes que d'habitude. La deuxième page est la page complète qui permet aux utilisateurs de changer leur mot de passe
  
  \vspace{0.7cm}
   
\hspace*{-0.7in}

               \noindent\makebox[\textwidth]{\includegraphics[width=\paperwidth,height=10cm]{img/password.png}}  
               
 
  Les mots de passe étaient erronés, la boîte d'alerte s'affichera comme ci-dessous :
  
   
  \vspace{0.7cm}
   
\hspace*{-0.7in}

               \noindent\makebox[\textwidth]{\includegraphics[width=\paperwidth,height=8cm]{img/password-wrong.png}}  
               
  
  
  Si vous entrez correctement votre mot de passe mais que les nouveaux mots de passe ne correspondent pas
  \vspace{0.7cm}
   
\hspace*{-0.7in}

               \noindent\makebox[\textwidth]{\includegraphics[width=\paperwidth,height=8cm]{img/password-matching.png}}  
               
  
  
  
  
\item \textcolor{red}{\huge Les problèmes rencontrés dans mon projet} :
\begin{enumerate}
\item    \textcolor{green}{ GMAIL et PHP} :

   \vspace{1.2cm}
                \setlength{\parindent}{1cm} Je n'ai pas réussi à effectuer certaines tâches complètement. Concernant le formulaire de contact, \textcolor{blue}{\textbf{GMAIL}} bloque tout message envoyé par \textcolor{blue}{\textbf{PHP}}, elles sont traités comme du spam. Mais l'application \textcolor{blue}{\textbf{PHPMAILER}} n'ai pas réussi à envoyer des mails à mon compte  comme le montre l'image suivante :
   
   \vspace{0.6cm}
   \hspace*{-1.05in}
               \noindent\makebox[\textwidth]{\includegraphics[width=\paperwidth,height=12cm]{img/gmail-block.PNG}}
  
    
    
    \item    \textcolor{green}{ XAMPP et HTACCESS} :
    
    \vspace{0.4cm}
                \setlength{\parindent}{1cm} Pour crypter les fichiers et dossiers qui contiennent des informations privées sur mon site Web, j'avais besoin d'utiliser le fichier \textcolor{blue}{\textbf{htaccess}}. Le fichier \textcolor{blue}{\textbf{htaccess}} est détecté et exécuté par le logiciel Apache Web Server, mais dans mon cas, \textcolor{blue}{\textbf{XAMPP}} ne peut pas exécuter le fichier \textcolor{blue}{\textbf{htaccess}}. L'image suivante montre l'erreur obtenue lors de la tentative d'entrer dans un fichier ou un dossier crypté :
    
    \vspace{0.6cm}

 \hspace*{-1.05in}
               \noindent\makebox[\textwidth]{\includegraphics[width=\paperwidth,height=9cm]{img/ht-error.PNG}}
  
   \vspace{0.6cm}  
	\warning Comme vous le voyez sur l'image même si j'entre le nom d'utilisateur et le mot de passe correct le serveur n'arrive pas à me laisser entrer dans ce dossier.  
  \end{enumerate}
\end{enumerate}


\end{document}
