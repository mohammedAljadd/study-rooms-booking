\documentclass{article}
\usepackage[utf8]{inputenc}
\usepackage{graphicx}
\usepackage[dvipsnames]{xcolor}
\definecolor{amethyst}{rgb}{0.6, 0.4, 0.8}
\title{Reservation des salles}
\author{Mohammed AL JADD}
\date{2020 Avril}
\begin{document}
\maketitle

    



\includegraphics[width=\textwidth]{img/Go.png}

\begin{enumerate}
    
     \item  \textcolor{red}{\huge Introduction} : 
     \vspace{1cm}
 
   \setlength{\parindent}{1cm} Mon projet est une application web , sous le nom Réservation des salles à distance, qui répondra aux problématiques suivantes:
   \begin{itemize}
     
        \item Difficulté à trouver une salle disponible.
        \item Le professeur a besoin d'être dans l'institut pour trouver une salle disponible.
        \item L'emploi du temps doit être modifié chaque fois qu'il n'y a pas de salle disponible.
        
            Les langages de programmation que j'ai utilisés dans mon projet : \textcolor{blue}{ \textbf{HTML}},\textcolor{blue}{\textbf{CSS}},\textcolor{blue}{\textbf{JavaScript}} et \textcolor{blue}{\textbf{PHP}}.
   \end{itemize}
   
   Il m'a fallu environ deux mois pour terminer ce projet. J'ai commencé en février et terminé en avril.\par Ce projet m'a permis d'apprendre à travailler avec github et à écrire du code long en \textcolor{blue}{\textbf{PHP}}, \textcolor{blue}{\textbf{CSS}} et \textcolor{blue}{\textbf{JavaScript}}. De plus, comme j'ai appris certaines commandes \textcolor{blue}{\textbf{GITHUB}}, cela me permettra à l'avenir de collaborer avec d'autres peuples, même à la maison.\par l'une des choses importantes que le projet me fournit est de savoir comment gérer votre projet et le discrétiser en étapes pour faciliter son progression.\par Pour \textcolor{blue}{\textbf{GITHUB}}, la meilleure chose que j'ai aimé est que vous ne serez pas attiré par votre machie puisque votre projet est stocké en ligne.
   
   
   
   
   
   
   \item \textcolor{red}{\huge L'avancement du projet} :
   
   
   
   \item \textcolor{red}{\huge L'illustration du calendrier du projet} :  
   
   \item \textcolor{red}{\huge La presentation du projet} :  
   \vspace{1cm}
   

   \item \textcolor{red}{\huge Les problèmes rencontrés dans mon projet} :
   
  
\end{enumerate}


\end{document}

